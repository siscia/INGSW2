% !TEX TS-program = pdflatex
% !TEX encoding = UTF-8 Unicode

% This is a simple template for a LaTeX document using the "article" class.
% See "book", "report", "letter" for other types of document.

\documentclass[11pt]{article} % use larger type; default would be 10pt.
\setcounter{secnumdepth}{2}

\usepackage{paralist} % very flexible & customisable lists (eg. enumerate/itemize, etc.)

\usepackage[utf8]{inputenc} % set input encoding (not needed with XeLaTeX)
\usepackage{float} % to place float images correctly
\usepackage{color} % to color text
\usepackage{enumitem} % for lists
\usepackage{subfigure} % for mockups
\usepackage[font={it}]{caption} % for captions

%%% Examples of Article customizations
% These packages are optional, depending whether you want the features they provide.
% See the LaTeX Companion or other references for full information.

%%% PAGE DIMENSIONS
\usepackage{geometry} % to change the page dimensions
\geometry{a4paper} % or letterpaper (US) or a5paper or....
% \geometry{margin=2in} % for example, change the margins to 2 inches all round
% \geometry{landscape} % set up the page for landscape
%   read geometry.pdf for detailed page layout information

\usepackage{graphicx} % support the \includegraphics command and options

% \usepackage[parfill]{parskip} % Activate to begin paragraphs with an empty line rather than an indent

\usepackage{listings}
\usepackage{color}
 
\definecolor{codegreen}{rgb}{0,0.4,0}
\definecolor{codegray}{rgb}{0.5,0.5,0.5}
\definecolor{codepurple}{rgb}{0.58,0,0.82}
 
\lstdefinestyle{mystyle}{ 
    commentstyle=\color{magenta},
    keywordstyle=\color{blue}\bfseries,
    numberstyle=\tiny\color{codegray},
    stringstyle=\color{codepurple},
    basicstyle=\footnotesize,
    breakatwhitespace=false,         
    breaklines=true,                 
    captionpos=b,                    
    keepspaces=true,                 
    numbers=left,                    
    numbersep=5pt,                  
    showspaces=false,                
    showstringspaces=false,
    showtabs=false,                  
    tabsize=2,
   emph={self}, 
   emphstyle=\color{blue}, 
   emph={[2] BookingManager, FineManager}, 
   emphstyle=[2]\color{codegreen}\bfseries, 
   emph={[3]__init__, newBook, removeReservation, getReservation, manageExpired, min_heap, pop, now, timedelta, expireReservation}, 
   emphstyle=[3]\color{codegreen}, 
}
 
\lstset{style=mystyle}





%%% PACKAGES
\usepackage{booktabs} % for much better looking tables
\usepackage{array} % for better arrays (eg matrices) in maths
%\usepackage{paralist} % very flexible & customisable lists (eg. enumerate/itemize, etc.)
\usepackage{verbatim} % adds environment for commenting out blocks of text & for better verbatim
\usepackage{subfig} % make it possible to include more than one captioned figure/table in a single float
% These packages are all incorporated in the memoir class to one degree or another...

%%% HEADERS & FOOTERS
\usepackage{fancyhdr} % This should be set AFTER setting up the page geometry
\pagestyle{fancy} % options: empty , plain , fancy
\renewcommand{\headrulewidth}{0pt} % customise the layout...
\lhead{}\chead{}\rhead{}
\lfoot{}\cfoot{\thepage}\rfoot{}

%%% SECTION TITLE APPEARANCE
\usepackage{sectsty}
\allsectionsfont{\sffamily\mdseries\upshape} % (See the fntguide.pdf for font help)
% (This matches ConTeXt defaults)

%%% ToC (table of contents) APPEARANCE
\usepackage[nottoc,notlof,notlot]{tocbibind} % Put the bibliography in the ToC
\usepackage[titles,subfigure]{tocloft} % Alter the style of the Table of Contents
\renewcommand{\cftsecfont}{\rmfamily\mdseries\upshape}
\renewcommand{\cftsecpagefont}{\rmfamily\mdseries\upshape} % No bold!

\newcommand{\pe}{PowerEnJoy }
\newcommand{\pecomma}{PowerEnJoy, }
\newcommand{\bul}[1]{\indent$\bullet$ #1\\}

\newcommand{\toBeIntegrated}[4]{
\begin{description}
	\item[Component1]: #1
	\item[Component2]: #2
	\item[Functionality]: #3
	\item[Description]: #4
\end{description}}

\newcommand{\toBeIntegratedThree}[5]{
\begin{description}
	\item[Component1]: #1
	\item[Component2]: #2
	\item[Component3]: #3
	\item[Functionality]: #4
	\item[Description]: #5
\end{description}}

\newcommand{\toBeIntegratedFour}[6]{
\begin{description}
	\item[Component1]: #1
	\item[Component2]: #2
	\item[Component3]: #3
	\item[Component4:] #4
	\item[Functionality]: #5
	\item[Description]: #6
\end{description}}



\usepackage{listings}
\usepackage{pxfonts}

\newcounter{ItegrationTest}
\newenvironment{test}[1][]{
	\refstepcounter{ItegrationTest}
	\item[Integration] IT {\ifnum\value{ItegrationTest}<10 0\fi \theItegrationTest}
	\item[Tests] \hfill 
		\begin{enumerate} #1}{
	\end{enumerate}}

\newcommand{\testCase}[2]{
\item \hfill 
\begin{description}
	\item[State of the system / Input] #1
	\item[Expected behaviour] #2
\end{description}}

%%% END Article customizations

%%% The "real" document content comes below...




\title{Integration	Test Plan Document}
\author{Simone Mosciatti \& Sara Zanzottera}

\begin{document}
\maketitle
\newpage
\tableofcontents
\newpage


\section{Introduction}

\subsection{Purpose}

\textit{[New purpose]}

\subsection{Scope}

\pe is a digital management system for car sharing that exclusively employs electric cars to provide its service. The system provides all the functionalities normally provided by a car sharing service: registering to the service, find the location of nearby available cars, reserve cars up to a short amount of time, unlock the chosen car once found, ride it and then park it in a safe area, when it will be automatically locked and the fee paid.

In addition, the system gives bonuses and penalities in term of discounts or overprices depending on the behavior of the user, in order to promote virtuous behaviors.

\pe is therefore a inherently distributed system, based on a central server interactions with many distributed nodes. All these components will be examined in more detail in the subsequent sections of the document.


\subsection{Definitions, Acronyms, Abbreviations}
 \begin{description}
	\item[RASD] Requirements and Specification Document.
	\item[DD] Design Document.
	\item[User] A customer of \pe using the service.
	\item[Staff Operator (Operator)] An employee of \pe which takes care of the cars.
	\item[Ride] The action of getting onboard of a \pe car, start its engine, drive to destination and park.
	\item[Issue] Any problem a car may incur in, or a user may face while using the service.
	\item[Nearby Cars] Available cars located within a maximum distance to a specific position.
	\item[Available Cars] Cars whose Availability Status is set to ``Available``.
	\item[Nearby Issues] Issues that are affecting cars close to a specific position.
	\item[Booking (Reservation)] The act to reserve a car for a limited amount of time for future use by a user.
	\item[Driver] Whoever is driving a regularly booked \pe car.
	\item[Driving License] The state's issued driving license of the user.
	\item[Notification] A form of comunication where the user is actively notified of some event.
	\item[Issue Report] An incoming notification that states a car incurred in an issue.
	\item[Fine] A fine issued by the local law enforcing officers to a user while driving a \pe car. 
	\item[Pending Bills] Bills that an user still need to pay to \pe.
	\item[Safe Area] An parking area, predefined by the company, where is possible to safely park the cars of the \pe fleet.
	\item[Battery Charge] The amount of charge that is kept inside the car's battery.
	\item[Charging Station (Power Station)] Dedicated areas where is possible to plug the \pe cars to charge their batteries.
	\item[Car's Onboard System] The controll system of the car that is able to exchange data with the central system and to relevate operation parameters.
	\item[Customer's App] An implementation of the system frontend tailored to the need of the customers.
	\item[Staff's App] An implementation of the system frontend tailored to the need of the staff.
	\item[Central System (Main Server)] The central system for \pe. All the command and all the data are streamed, analyzed and used here.
	\item[GPS]: Global Positioning System is a global navigation satellite system (GNSS) that provides location and time information in all weather conditions, anywhere on or near the Earth where there is an unobstructed line of sight to four or more GPS satellites.
	\item[Location] Pair of integer values as provided by GPS sensors.
	\item[Payment Method] Set of data relative to a credit card.
	\item[Email address (Email)] Unique string identifiying an email box to which email messages are delivered.
	\item[Identity ID] Personal code provided by local authorities to uniquely identify citizens.
	\item[Driving License ID] The unique code reported on every legal driving license.
	\item[Session Key]  A string representing a key for a secured channel. Used to secure communications between the server and the nodes.
	\item[Scanned License] An high quality image of the driving license acquired by the car's onboard system.
  \end{description}

\subsection{Reference Documents}
\begin{itemize}
	\item \textit{Assignments AA 2016-2017.pdf} (Assignments document given by the teacher)
	\item \textit{Reqiurements And Specification Document} (referring to this project)
	\item \textit{Design Document} (referring to this project)
  \end{itemize}



\newpage
\section{Integration Strategy}

We believe that the integration of different components should not be a completely different phase from the development of each component.

Integration should happens as soon as possible even with mock implementation of the components, such early integration helps in identify possible inconsistency in the design documents and the sooner those inconsistencies are discovered the simpler will be to fix them.

\subsection{Entry criteria}

As long as the integration test are automatic and fast we do not advocate for any strong or specific precondition.

On the other side we advocate to run the integration test after each commit, this will help in catching as early as possible regressions.

Unit test should be used in order to guarantee the interface exposed and documented are respected.

\subsection{Elements to be integrated}

Clearly all the components that interact with the user must be integrated with the User FrontEnd, similarly all the components that need to interact with the Staff need to be integrated with the Staff FrontEnd.

Also, the GEOLOCATION component need to be integrated with the Car GUI.

Concerning external services only few components need to interact with them, in particular we need to integrate the BILLING\_SYSTEM, POSITION, GEOLOCATION, USER\_MANAGER.

Finally, few services must be integrated between each other in the system thanks to the high degree of decoupling reached during the design phase.

The services that must be integrated inside the system are the following:

\begin{description}

\item[IT01] \hfill
\toBeIntegrated
	{GEOLOCATION}
	{POSITION}
	{GEOLOCATION/AvailableCar}
	{The position of each car, retrieved using POSITION/Car is used in order to describe what car is close to a specific position}

\item[IT02] \hfill
\toBeIntegratedThree
	{GEOLOCATION}
	{POSITION}
	{CAR\_MANAGER}
	{GEOLOCATION/Issues}
	{The position of each car, retrieved using POSITION/CAR that has some issues is, detecatable using CAR/Telemetry is used to communicate to the staff where are the cars that need attention.}

\item[IT03] \hfill
\toBeIntegrated
	{RIDE\_MANAGER}
	{CAR\_MANAGER}
	{RIDE/Start}
	{When a ride start, detected using CAR/Telemetry, the RIDE/Start method is invoked}

\item[IT04] \hfill
\toBeIntegrated
	{RIDE\_MANAGER}
	{CAR\_MANAGER}
	{RIDE/End}
	{When a ride stop, detected using CAR/Telemetry, the RIDE/End method is invoked}

\item[IT05] \hfill
\toBeIntegrated
	{BILLING\_MANAGER}
	{RIDE\_MANAGER}
	{BILL/Pay}
	{At the end of a ride, detected because the procedure RIDE/End is been called, it start the procedure to receive the payment from the user.}

\item[IT06] \hfill
\toBeIntegrated
	{BILLING\_MANAGER}
	{BOOKING\_MANAGER}
	{BILL/CalculateExpireBookFee}
	{When a booking expire, detected using BOOKING/Expire, the user should then be asked to pay the fee calculated using BILL/CalculateExpireBookFee}
	
\item[IT07] \hfill
\toBeIntegratedFour
	{BILLING\_MANAGER}
	{RIDE\_MANAGER}
	{POSITION}
	{GEOLOCATION}
	{BILL/CalculateUnsafeParkingFine}
	{When the user ends a ride, detected using RIDE/End, and the car is unsafely parked, detected using POSITION/Car and GEOLOCATION/IsSafeArea, the user will be prompted to pay for the ride plus the fee for having park the car in an unsafe position.}
	
\item[IT08] \hfill
\toBeIntegrated
	{BILLING\_MANAGER}
	{CAR\_MANAGER}
	{BILL/CalculateRideFee}
	{After a ride is concluded the user is asked to pay is determinated on the base of several bonuses that may be applied, to determinate which bonuses apply we use CAR/Telemetry}

\end{description}

\subsection{Integration testing strategy}

We will start integrating the three services that are logically the same but physically distributed, in this way we will be able to reason about them as a simpler logical component instead of multiple distributed components.

\begin{enumerate}
	\item CAR\_MANAGER, this component need its own integration, it is distributed between the car system and the main server, it is also the only components that touch the hardware side of the project, in our experience it should be the first component to be tested and be well defined.
	\item ISSUE\_MANAGER, another component that needs it own integration, however it functionality are smaller that the one provided by the CAR\_MANAGER.
	\item POSITION, also this component need it own integration, however its functionality are pretty standard and well know.
\end{enumerate}

In parallel we can proced to integrate the external services with our own.

We will finish integrating the ``internal'' serviceses.

Of course, as soon as a component is ready and integrated it will be integrated with the User Interfaces.

\subsection{Sequence of components integration}

Our design tried to keep the components very decoupled, this allow us a lot of flexibility in the integration process.

There are not hard constraints in what component should be integrated first.

If a functionality need the integration of some components and such integration is not yet complete, simply that functionality won't be ready.

Of course all the components need to be integrated with the user interface, we are not going to show all those integration that however are extremely simple but necessary. Since the NOTIFIER components is part of the user interface, we are not going to show the integration that concern this component.

Before to go deeply inside the integration, we show what integration are necessary in order to treat all the components as a single logical entity, instead of multiple, more complex physical entities.

\begin{figure}[H]
	\centering
	\includegraphics[width=1.1\textwidth]{UML/IntegrationDistributedComponents.png}
	\caption{Integration of the physical splitted components into single logical componets.}
\end{figure}	

In the following diagram we show what components integrates with others.

There are not strict dependencies, and it is more a net that a pyramid.

\begin{figure}[H]
	\centering
	\includegraphics[width=1.1\textwidth]{UML/IntegrationStrategy.png}
	\caption{Integration net}
\end{figure}	


\newpage
\section{Individual Steps and Tests Description}

In this section we describe what test are necessary for each integration.

For each integration we are going to show the most important tests.

\begin{description}
	\begin{test}
		\testCase{An available car is on coordinates A}{Each invocation of GEOLOCATION/AvailableCar that include the coordinates A show the car.}
		\testCase{No available car is inside a circle of radius R centered in the point C}{Each invocation of GEOLOCATION/AvailableCar that is centered in the point C with radius smaller of equal than R returns an empty set.}
	\end{test}
	\begin{test}
		\testCase{A car with issues determinated using the onboard system is located on point A}{Each invocation of GEOLOCATION/Issue that include the point A show the car.}
		\testCase{No car with any kind of issues are inside a circle of radius R centered in point C}{Each invocation of GEOLOCATION/Issue that is centered in point C of radius less or equal than R returns an empty set.}
	\end{test}
	\begin{test}
		\testCase{The signal of a ride started is send by the CAR/Telemetry}{The method RIDE/Start is invoked and succeed}
	\end{test}
	\begin{test}
		\testCase{The signal of a ride finished is send by the CAR/Telemetry}{The method RIDE/End is invoked and succeed}
	\end{test}
	\begin{test}
		\testCase{The method RIDE/End is invoked}{The method BILL/Pay is invoked and it start the whole procedure to get the user to pay for the ride.}
	\end{test}
	\begin{test}
		\testCase{The method BOOKING/Expire is invoked}{The method BILL/CalculateExpireBookFee is invoked and it is used to provide the ammout of money the user own}
	\end{test}
	\begin{test}
		\testCase{The method RIDE/End is invoked, the car position (determinate using POSITION/Car) is not inside a safe area (determinate using GEOLOCATION/IsSafeArea)}{The method BILL/CalculateUnsafeParkingFine is invoked and it is used to provide the ammount of money that the user must pay.}
	\end{test}
	\begin{test}
		\testCase{The method RIDE/End is invoked, the car is safely parked}{The method BILL/CalculateRideFee is invoked, the method use all the necessary information from the CAR\_MANAGER to apply all the bonuses or the maluses.}
	\end{test}
\end{description}

This test are obviously only a guidelines. We have describe only the most straightforward test that must be implemented, however, other test can be deployed too.

Usually it is a good idea to test, not only that a method is invoked and used, but that a method is not invoked. We do not specifically describe those tests because they are extremely implementation dependent, in some implementation that kind of test is extremely important, in other is quite useles.

\newpage
\section{Tools and Test Equipment Required}

Any tools or framework will work just as well in this project.

Our suggestion is to use a well know libraries and framework.

A very reasonable choice is JUnit4, which is the default testing framework for Java application.

\subsection{Continuos Integration and Unit Testing}

A server of continuos integration will be necessary in order to execute the integration tests and the  unit tests after each commit to the main code base.

Each developer should still be able to run every test in its developing machine.

The unit test suite is expected to be extremelly fast so that it can be executed at will by every developer without slowing down their workflow.

For the integration test, we don't require them to be fast, although is still a very desiderable property, but they should at very least be completely automatic.

\subsection{Testing Strategy}

We do not advocate for rigid methodology. 

Test should be produced in order to avoid regression. 

After fixing a bug a test relative to that same bug should be put in place and should be continuosly run.

\subsection{Deploying}

The deploy of the whole architecture should require the least ammount of step possible, ideally just one.

If the application is serving life traffic a deploy of a new version should be done following the incremental GREEN / BLUE methodology. Start direct only a small fraction of the overall traffic to the new version and if everything on the control metrics still looks fine start to increase the ammount of traffic to the new version untill you don't reach the totality. Immediately roll back to the previous version at the first sign of problem.

\newpage
\section{Program Stubs and Test Data Required}

\subsection{Stubs}

The use of stubs makes definitely sense when we want to simulate the physical car, we definitely cannot have a car physicall to run all the integration test.

Also, the developing time of the hardware are usualy longer than the developing time of software and it is reasonable to use stumbs to simulate the hardware of the car.

Other stumbs may be required to develop the integration test and will be written when necessary.

\subsection{Test Data}

The test data necessary will be generated using ad hoc software.

There will be two different databases, one for tests and another for production, if it is necessary to simulate the load of the production database, we will not directly use it, but we will populate the test database.

\newpage
\section{Conclusions}

\subsection{Tools used}
During the development of this document we used the following tools:
\begin{itemize}
	\item \textbf{Github} to version control the project
	\item \textbf{\LaTeX} on TeXworks to redact this document
\end{itemize}

\subsection{Hours of work}
\begin{itemize}
	\item SZ: 1h on 21/12
\end{itemize}


\end{document}  
