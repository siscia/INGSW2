% !TEX TS-program = pdflatex
% !TEX encoding = UTF-8 Unicode

% This is a simple template for a LaTeX document using the "article" class.
% See "book", "report", "letter" for other types of document.

\documentclass[11pt]{article} % use larger type; default would be 10pt.
\setcounter{secnumdepth}{2}

\usepackage{paralist} % very flexible & customisable lists (eg. enumerate/itemize, etc.)

\usepackage[utf8]{inputenc} % set input encoding (not needed with XeLaTeX)
\usepackage{float} % to place float images correctly
\usepackage{color} % to color text
\usepackage{enumitem} % for lists
\usepackage{subfigure} % for mockups

%%% Examples of Article customizations
% These packages are optional, depending whether you want the features they provide.
% See the LaTeX Companion or other references for full information.

%%% PAGE DIMENSIONS
\usepackage{geometry} % to change the page dimensions
\geometry{a4paper} % or letterpaper (US) or a5paper or....
% \geometry{margin=2in} % for example, change the margins to 2 inches all round
% \geometry{landscape} % set up the page for landscape
%   read geometry.pdf for detailed page layout information

\usepackage{graphicx} % support the \includegraphics command and options

% \usepackage[parfill]{parskip} % Activate to begin paragraphs with an empty line rather than an indent

%%% PACKAGES
\usepackage{booktabs} % for much better looking tables
\usepackage{array} % for better arrays (eg matrices) in maths
%\usepackage{paralist} % very flexible & customisable lists (eg. enumerate/itemize, etc.)
\usepackage{verbatim} % adds environment for commenting out blocks of text & for better verbatim
\usepackage{subfig} % make it possible to include more than one captioned figure/table in a single float
% These packages are all incorporated in the memoir class to one degree or another...

%%% HEADERS & FOOTERS
\usepackage{fancyhdr} % This should be set AFTER setting up the page geometry
\pagestyle{fancy} % options: empty , plain , fancy
\renewcommand{\headrulewidth}{0pt} % customise the layout...
\lhead{}\chead{}\rhead{}
\lfoot{}\cfoot{\thepage}\rfoot{}

%%% SECTION TITLE APPEARANCE
\usepackage{sectsty}
\allsectionsfont{\sffamily\mdseries\upshape} % (See the fntguide.pdf for font help)
% (This matches ConTeXt defaults)

%%% ToC (table of contents) APPEARANCE
\usepackage[nottoc,notlof,notlot]{tocbibind} % Put the bibliography in the ToC
\usepackage[titles,subfigure]{tocloft} % Alter the style of the Table of Contents
\renewcommand{\cftsecfont}{\rmfamily\mdseries\upshape}
\renewcommand{\cftsecpagefont}{\rmfamily\mdseries\upshape} % No bold!

\newcommand{\pe}{PowerEnJoy }
\newcommand{\pecomma}{PowerEnJoy, }
\newcommand{\bul}[1]{\indent$\bullet$ #1\\}

\usepackage{listings}
\usepackage{pxfonts}
%%% END Article customizations

%%% The "real" document content comes below...




\title{Design Document}
\author{Simone Mosciatti \& Sara Zanzottera}

\begin{document}
\maketitle
\newpage
\tableofcontents
\newpage


\section{Introduction}

\subsection{Purpose}

This Design Document aims to provide to everyone involved in the actual development of the application specific insights about the structure of \pecomma its acthitecture's details, the desing patterns we chosed to implement, but also some details about its high level components, their interactions and general behavior.

\subsection{Scope}

\pe is a digital management system for car sharing that exclusively employs electric cars to provide its service. The system provides all the functionalities normally provided by a car sharing service: registering to the service, find the location of nearby available cars, reserve cars up to a short amount of time, unlock the chosen car once found, ride it and then park it in a safe area, when it will be automatically locked and the fee paid.

In addition, the system gives bonuses and penalities in term of discounts or overprices depending on the behavior of the user, in order to promote virtuous behaviors.

\pe is therefore a inherently distributed system, based on a central server interactions with many distributed nodes. In detail the system can be divided into four main parts: 

\begin{itemize}[noitemsep]
	\item a public app, used by customers to access the service
	\item a centralized backend that provides the service
	\item the cars' onboard system, that communicates only with the centralized backend
	\item a reserved fronted, used exclusively by the staff members to better organize their job
\end{itemize}
All these four components will be examined in more detail in the subsequent sections of the document.


\subsection{Definitions, Acronyms, Abbreviations}
 \begin{description}
	\item[RASD] Requirements and Specification Document.
	\item[DD] Design Document.
	\item[User] A customer of \pe using the service.
	\item[Staff Operator] An employee of \pe which takes care of the cars.
	\item[Ride] The action of getting onboard of a \pe car, start its engine, drive to destination and park.
  	\item[Running Time] The time an user spends using the \pe service.
	\item[Issue] Any problem a car may incur in, or a user may face while using the service.
	\item[Nearby Cars] Cars located within a maximum distance to a specific position.
	\item[Nearby Issues] Issues that are affecting cars close to a specific position.
	\item[Booking (Reservation)] The act to reserve a car for a limited amount of time for future use by a user.
	\item[Reservation's maximum time] The maximun amount of time a car can be reserved.
	\item[Driver] Whoever is driving a regularly booked \pe car.
	\item[Passenger] Whoever is in inside a \pe car but is not the driver.
	\item[Driving License] The state's issued driving license of the user.
	\item[Notification] A form of comunication where the user is actively notified of some event.
	\item[Issue Report] An incoming notification that states a car incurred in an issue.
	\item[Fine] A fine issued by the local law enforcing officers to a user while driving a \pe car. 
	\item[Pending Bills] Bills that an user still need to pay to \pe.
	\item[Safe Area] An parking area, predefined by the company, where is possible to safely park the cars of the \pe fleet.
	\item[Battery Charge] The amount of charge that is kept inside the car's battery.
	\item[Charging Station] Dedicated areas where is possible to plug the \pe cars to charge their batteries.
	\item[Car's Onboard System] The controll system of the car that is able to exchange data with the central system and to relevate operation parameters.
	\item[Customer's App] An implementation of the system frontend tailored to the need of the customers.
	\item[Operator's App] An implementation of the system frontend tailored to the need of the staff.
	\item[Central System] The central system for \pe. All the command and all the data are streamed, analyzed and used here.
	\item[Credentials] Pair \{Username, Password\} necessary to access the \pe system.
  	\item[GPS]: Global Positioning System is a global navigation satellite system (GNSS) that provides location and time information in all weather conditions, anywhere on or near the Earth where there is an unobstructed line of sight to four or more GPS satellites.
  	\item[System's Frontend] The interface provided to the user of the \pe system. 
  	\item[System's Backend]  The whole technical infrastructure necessary to \pe.
  \end{description}

\subsection{Document Structure}

\begin{enumerate}
	\item \textbf{Introduction}

	This sections aims to explain the purpose and the scope of the document, introducing the reader to subsequent sections of the document itself.

	\item \textbf{Architectural Design}
	 
	This sections will explain the main architectural decision we made.

	\item \textbf{Algorithm Design}

	In this section we focus on the most critical code section and we provide an in-depth analysis of how they should be structured, eventually providing pseudocode for them.
	
	\item \textbf{User Interface Design}

	In this section we carry on the UX design with the help of UX and BCE diagrams, eventually completing them with updated and extended application mockups.

	\item \textbf{Requirements Traceability}
	
	In this section we map the requirements stated in the RASD to the actual component or processes that fulfill these requirements.

	\item \textbf{Conclusions}

	In this section we enumerate the tools we used to redact this document, the hours of work spent by each group member and the (eventual) revision history of the document itself.
\end{enumerate}

\subsection{Reference Documents}
\begin{itemize}
	\item \textit{Assignments AA 2016-2017.pdf} (Assignments document given by the teacher)
	\item \textit{Sample Design Deliverable Discussed on Nov. 2.pdf} (Sample document provided by the teacher)
  \end{itemize}




\newpage
\section{Architectural Design}
The overall design process has been carried in a bottom-up approach, starting from the analysis of goals and requirements moving upwards to the definition of the higher level components of the system. In the following sections we provide more details on the designed architecture.

\subsection{Design Process Description}

The overall design process starts from the physical structure of the system. Taking forward the considerations made in the RASD, we identify which physical nodes that are to be deployed in order to meet the requirements, and make the very first design choices onto them.

Then we analyse the interface between the world and the machine, according to our choices. Given the list of goals and requirements, we identify what interfaces each node should provide to the world and to other nodes in order to accomplish such goals.

Once the interfaces are identified, we proceed organizing those interfaces into higher-level components, caring at respecting the Single Responsability principle in order to provide highly decoupled and reusable components.

Once components have been defined, we organize them into an high level architecture and define in detail how they interact with the help of some Sequence Diagrams.

The rest of this section follows this flow, starting from the deploy analysis and finally providing the overall design.



\subsection{System's Structure}

Asfor the RASD (section 2: Overall Description), the system is be divided into four parts:
\begin{itemize}[noitemsep]
	\item the \textbf{customer's app}, used by customers to access the service.
	\item the \textbf{main server}, a centralized backend that provides the service.
	\item the \textbf{cars’ onboard system}, that communicates only with the centralized backend.
	\item the \textbf{staff's app}, used exclusively by the staff members to better organize their job.
\end{itemize}

For the scope of this analysis we often consider the customer's app and the staff's app as a single entity, called simply ``app``.

At a first glance, the above elements can be organized in a two-tier Client-Server architecture as follows:
\begin{itemize}[noitemsep]
	\item Tier 1, the main server, which handles Application Logic and Data Management.
	\item Tier 2, comprising mobile apps and cars, hosts the User Interface.
\end{itemize}
From now on, we design the system basing on these fundamentals elements, defining their roles and interactions.




\subsection{Main Design Choices}

Our choices are, first of all, focused on enforcing the architecture identified above. 

\subsubsection{Car to app communication}
We decide to completely avoid all kinds of communication between the cars and the apps. 

 The main server is always an intermediary for app-to-car and car-to-app communications, in order for the system to have always full control on these otherwhise completely hidden interactions.

\subsubsection{Interactions with 3rd-parties}
We decided to prevent any access to third-part services to every component but the main server, mainly for security reasons.

The main server will expose the necessary API to apps and cars to allow them retrieving all the informations they need without having them communicating indipendently with any external service.

\subsubsection{Communication protocols}
More specific choices has been made on the communication protocols too.

A pure \textbf{Client-Server} communication protocol is implemented only for server-to-app and app-to-server communications, as they seem to naturally fit this model.

On the other hand, communications between the server and the fleet is clearly more suitable for a \textbf{Publisher-Subscriber} pattern. The car has to communicate very often a lot of valuable information to the main servet, and a PubSub pattern achieves high troughput and low latency.
 
To describe the communication approach in terms of the protocol itself, the car publishes messages about it own status and the server subscribes to those messages. The server is meant to host brokers too.

Moreover, we decided to model the most part of external services as Web services, so the server will communicate with them using a \textbf{Service-Oriented} approach.

\hfill\

Here we provide a high-level diagram of the system and some proposed technologies that we will discuss in the last sections of  the document.


\begin{figure}[H]
	\centering
	\includegraphics[width=1\textwidth]{proposed_system.png}
	\caption{High-level organization of the system's required elements}
\end{figure}





\subsection{Component Interfaces}

We now proceed defining the API that the nodes expose to each other, mapping them to the requirements they fulfill. Note that we are defining only internal interfaces: interface that components are exposing to the user are specified in the subsequent section ``User Interfaces``.

In interfaces definitions, we reported only requirement's codes. For requirements definitions, see RASD section 3.2: Functional Requirements.

\subsubsection{Server to All Apps API }
List of interfaces exposed by the server to both customer's and staff's apps.
\begin{description}
	\item[LOGIN] Provided valid credentials, users are logged into the system. \\ Fulfills \textbf{LOG1, LOG2, LOG3} and \textbf{LOG4}
	\item[UNLOCK] Logged users can unlock the car they booked. \\ Fulfills \textbf{UNLK1, UNLK2, UNLK3, UNLK5, SUP5, SUP6}	
	\item[LOCK] Logged users lock their car. \\ Fulfills \textbf{RIDE5, SAFE3, SAFE4}
	\item[REPORT\_ISSUE] Users can report issues about a car. \\ Fulfills \textbf{REP1}
\end{description}


\subsubsection{Server to Customer's App API}
List of interfaces exposed by the server only to customer's apps.
\begin{description}
	\item[REGISTER] Users provide their personal informations, including license number and billing information, and obtain an account. \\ Fulfills \textbf{REG1}.
	\item[VALIDATE] The system validates the informations provided at registration time. \\ Fulfills \textbf{REG2} and \textbf{REG3}.
	\item[LOOKUP] Logged users can retrieve a list of available cars according to their search settings.\\ Fulfills \textbf{LOOK1, LOOK2} and \textbf{LOOK3} 
	\item[BOOK] Logged users can reserve a car. \\ Fulfills \textbf{BOOK2, BOOK3, BOOK4, BOOK5 }	
	\item[UNBOOK] Logged users can cancel a reservation they made. \\ Fulfills \textbf{UNBOOK2}	
	\item[CALCULATE\_FEE] The system calculates the total fee that the user must pay when the car gets locked. \\ Fulfills \textbf{FEE1, FEE2, FEE3, FEE4, FEE5, FEE6}	
	\item[SET\_PAYMENT\_METHOD] Users can set their preferred paying method. \\ Fulfills \textbf{PAY1}
\end{description}


\subsubsection{Server to Staff's App API}
List of interfaces exposed by the server only to staff's apps.
\begin{description}
	\item[RIDE] The system cal tell which user is driving which car at the present time and at any moment in the past. \\ Fulfills \textbf{RIDE1}
	\item[FIND\_ISSUE] Staff operators can locate issued cars that needs their intervantion. \\ Fulfills \textbf{ISS1, ISS3}
	\item[TAKE\_CHARGE] The system allows operators to take charge of certain issues. \\ Fulfills \textbf{SUP1, SUP6}
	\item[SOLVE] The system allows the operator to mark an issue as solved. \\ Fulfills \textbf{SUP3}
	\item[GIVE\_UP] The system allows the operator to give up over an issue. \\ Fulfills \textbf{SUP7}
	\item[SET\_STATUS] Operators can change the Exception status of issued cars. \\ Fulfills \textbf{SUP4}
\end{description}


\subsubsection{Server to Car API}
List of interfaces exposed by the server only to the cars' onboard system.
\begin{description}
	\item[SHOW\_INFORMATIONS] Once the ride started, the car receives basic informations such as nearby safe parking areas and nearby charging stations. \\ Fulfills \textbf{RIDE4, SAFE1, SAFE2, PWRS1, PWRS2}
	\item[VALIDATE\_LICENSE] The system uses input information about the user's driving license to decide whether to grant the user permission to start the car's engine. \\ Fulfills \textbf{RIDE2, RIDE3, SUP6}
\end{description}


\subsubsection{Customer's App to Server API}
List of interfaces exposed by the customer's apps to the main server.
\begin{description}
	\item[EXPIRE] A booked car not unlocked after a system-defined period of time has passed is automatically unbooked and the user who booked it is fined. \\ Fulfills \textbf{BOOK6}
	\item[PAY] The app receive payment requests. \\ Fulfills \textbf{PAY4}
\end{description}


\subsubsection{Staff's App to Server API}
We identified no interfaces exposed by the staff's apps to the main server at this stage.


\subsubsection{Car to Server API}
List of interfaces exposed by the car's onboard system to the main server.

Note that the communication protocol between the server and the car is meant to be message-based: these API thus represent messages exchanged between the server and the car.
\begin{description}
	\item[ENGINE\_STATUS] The car reports the engine status (True if running, False otherwise). \\ Fulfills \textbf{UNSF2}
	\item[NUM\_PASSENGERS] The car reports how many passengers are onboard (excluding the driver).
	\item[DRIVER\_PRESENT] The car reports if the driver is inside the car (True) or not (False). \\ Fulfills \textbf{UNSF2}
	\item[PLUGGED] The car reports if it is connected to a charging station (True) or not (False). \\ Fulfills \textbf{PWRS3}
	\item[LOCKED] The car reports if it is locked (True) or unlocked (False). \\ Fulfills \textbf{UNSF2}
	\item[BATTERY\_LEVEL] The car reports its battery level (100 fully charged, 0 fully empty)
	\item[GPS\_DATA] The car reports its GPS coordinates as retrieved from the sensor. \\ Fulfills \textbf{UNSF2}
	\item[MOVING] The car reports if it is moving (True) or if it stopped (False), regardless of the engine status. \\ Fulfills \textbf{UNSF2}
	\item[ISSUES] The car reports its Exception Status (could be ``No Issue``).

	\item[LOCK\_CAR] The system forces the car to lock when left in an unsafe area. \\ Fulfills \textbf{UNSF3}
	\item[ENGINE\_OFF] The system forces the cas to switch its engine off when the car is left in an unsafe area.\\ Fulfills \textbf{UNSF4}
	\item[SET\_EXCEPTION\_STATUS] The system sets the car's Exception Status to ``Unsafely Parked`` if it understands it is parked in an unsafe area. \\ Fulfills \textbf{UNSF2}
	
	\item[VALIDATE\_SOLVE] The car performs a self-check and confirms to the server its Exception status can be set back to ``No Issue``
	\item[REPORTED\_ISSUE] The car sets its Avaliability status as ``Not Available`` and its Exception status as ``Other issue`` as the system receives a related issue report. \\ Fulfills \textbf{REP2}
\end{description}

\subsubsection{External Services to Server API}
List of interfaces exposed by external services to the main server. {\color{red} {TODO! }} 

{\color{blue} { Qui il diagramma delle interfacce a pagina 18 dell'esempio. Non capisco come faccia il loro a essere cosi magro...}}

\subsection{User Interfaces}

{\color{red} {TODO! }}

Now we define in details which interfaces are being exposed to the final user. These are going to be better defined later, but we report them here in order to show exactly which requirements are meant to satisfy.




\subsection{Component View}
Given the interface we identified in the previous section, we organize such interfaces into higher level components as follows.

Components are described as \textbf{COMPONENT\_NAME/Functionality\_Name}, then highlighting responsibility, input and output data, and related interfaces for each function.

All the components are responsible of returning meaningful error messages in case of error that we are not going to specify in detail.

\begin{description}
	\item[USER\_MANAGER] \hfill
	\begin{description}
		\item[Responsability] Manages the users.
	\item[USER/Register] \hfill
		\begin{description}[noitemsep]
			\item[Responsability] Registers a new user into the system.
			\item[Interfaces Implemented] REGISTER, VALIDATE.
			\item[Input] Information from the user such as:
				\begin{itemize}
					\item First name
					\item Last name
					\item Password
					\item Email
					\item License ID
					\item Credit card informations: credit card number, control code, expiry date, owner, etc.
				\end{itemize}
			\item[Output] The ID of the newly created user.
		\end{description}
	\item[USER/Login] \hfill
		\begin{description}[noitemsep]
			\item[Responsability] Allows users to log into the system.
			\item[Interfaces Implemented] LOGIN.
			\item[Input] Email (considered a unique user ID) and password.
			\item[Output] A session key, meaning that the user is logged into the system.
		\end{description}
	\item[USERL/SetPaymentMethod] \hfill
		\begin{description}[noitemsep]
			\item[Responsability] Update user's information about the preferred payment method.
			\item[Interfaces Implemented] SET\_PAYMENT\_METHOD.
			\item[Input] The ID of the user and new payment informations.
			\item[Output] The payment method data related to this user is updated.
		\end{description}
	\end{description}
	
	\item[LOCATION] \hfill
	\begin{description}
		\item[Responsability] Locates elements, points and areas of interest around a specific coordinate. ``Search`` service for elements of interest.
	\item[LOCATION/AvailableCar] \hfill
		\begin{description}[noitemsep]
			\item[Responsability] Retrives the position of available cars.
			\item[Interfaces Implemented] LOOKUP
			\item[Input] Search parameters such as:
			\begin{itemize}
				\item Geographical coordinates of the center of the search range (latitude and longitude as provided by GPS sensors) 
				\item Maximum walking distance from the specified position
				\item Other search settings, like minimum battery level, etc.
			\end{itemize}
			\item[Output] A set of available cars matching the search parameters.
		\end{description}

	\item[LOCATION/Areas] \hfill
		\begin{description} [noitemsep]
			\item[Responsability] Retrives the position of areas of interest, such as power stations and safe parking areas.
			\item[Interfaces Implemented] SHOW\_INFORMATIONS
			\item[Input] Geographical coordinates of the center of the search range (latitude and longitude as provided by GPS sensors) and a search radius.
			\item[Output] A set of areas of interest inside the circle of radius provided centered on the coordinates provided.
		\end{description}

	\item[LOCATION/Issues] \hfill
		\begin{description}[noitemsep]
			\item[Responsability] Retrives the position of cars with some issues.
			\item[Interfaces Implemented] FIND\_ISSUES.
			\item[Input] Search parameters such as:
			\begin{itemize}
				\item Geographical coordinates of the center of the search range (latitude and longitude as provided by GPS sensors) 
				\item Radius of the search
				\item Issue type, Exeption status, and other similar search settings.
			\end{itemize}
			\item[Output] A set of cars with issues matching the search parameters inside the circle of radius provided centered on the coordinates provided.
		\end{description}
	\end{description}

	\item[POSITION] \hfill
	\begin{description}
		\item[Responsability] Locatse elements of interest given their ID. ``Lookup`` service for elements of interest.
	\item[POSITION/Car] \hfill
		\begin{description}[noitemsep]
			\item[Responsability] Retrieves the position of a specific car.
			\item[Interfaces Implemented] {\color{red} { USER Interface! }}
			\item[Input] The ID of the car.
			\item[Output] The coordinates of the car.
		\end{description}
	\item[POSITION/User] \hfill
		\begin{description}[noitemsep]
			\item[Responsability] Retrieves the position of an user.
			\item[Interfaces Implemented] {\color{red} { ??? }}
			\item[Input] The ID of the user.
			\item[Output] The coordinates of the user.
		\end{description}
	\item[POSITION/Areas] \hfill
		\begin{description}[noitemsep]
			\item[Responsability]Retrieves the position of an area of interest.
			\item[Interfaces Implemented] {\color{red} { USER Interface! }}
			\item[Input] The ID of the area.
			\item[Output] The coordinates of the area as a set of boundary points.
		\end{description}
	\end{description}
	
	\item[BOOKING\_MANAGER] \hfill
	\begin{description}
		\item[Responsability] Manages reservations.
	\item[BOOKING/Book] \hfill
		\begin{description}[noitemsep]
			\item[Responsability] Books one available car.
			\item[Interfaces Implemented] BOOK
			\item[Input] The ID of the car and the ID of the user.
			\item[Output] The car is booked and the ID of the reservation is provided.
		\end{description}
	\item[BOOKING/Unbook] \hfill
		\begin{description}[noitemsep]
			\item[Responsability] Removes a reservation.
			\item[Interfaces Implemented] UNBOOK
			\item[Input] The ID of the user and the ID of the reservation.
			\item[Output] The reservation is cancelled.
		\end{description}
	\item[BOOKING/Expire] \hfill
		\begin{description}[noitemsep]
			\item[Responsability] Removes an expired reservation and fines the related user.
			\item[Interfaces Implemented] EXPIRE
			\item[Input] ID of the reservation.
			\item[Output] The reservation is cancelled and the user is fined.
		\end{description}	
	\end{description}
	
	\item[CAR\_MANAGER] \hfill
	\begin{description}
		\item[Responsability] Manages the iteractions between users and cars.
	\item[CAR/Unlock] \hfill
		\begin{description}[noitemsep]
			\item[Responsability] Unlocks the car.
			\item[Interfaces Implemented] UNLOCK
			\item[Input] The ID of the car and the ID of the user asking to unlock.
			\item[Output] The car is unlocked.
		\end{description}
	\item[CAR/ValidateLicense] \hfill
		\begin{description}[noitemsep]
			\item[Responsability] Confirms the scanned license is related to the user who booked the car.
			\item[Interfaces Implemented] VALIDATE\_LICENSE
			\item[Input] The scanned image of the driving license and the ID of the booking
			\item[Output] The car is unlocked.
		\end{description}
	\item[CAR/Lock] \hfill
		\begin{description}[noitemsep]
			\item[Responsability] Locks the car.
			\item[Interfaces Implemented] LOCK, LOCK\_CAR
			\item[Input] The ID of the car.
			\item[Output] The car is locked.
		\end{description}
	\item[CAR/TurnOff] \hfill
		\begin{description}[noitemsep]
			\item[Responsability] Turns off the engine of a car.
			\item[Interfaces Implemented] ENGINE\_OFF
			\item[Input] The ID of the car.
			\item[Output] The car is turned off.
		\end{description}
	\item[CAR/Telemetry] \hfill
		\begin{description}[noitemsep]
			\item[Responsability] Retrieves real-time, updated informations about a car.
			\item[Interfaces Implemented] ENGINE\_STATUS, NUM\_PASSENGERS, DRIVER\_PRESENT, PLUGGED, LOCKED, BATTERY\_LEVEL,  GPS\_DATA, MOVING, ISSUES
			\item[Input] The ID of the car.
			\item[Output] All the latest informations available about the car.
		\end{description}
	\item[CAR/SetStatus] \hfill
		\begin{description}[noitemsep]
			\item[Responsability] Sets the Exception status of a car to a new value.
			\item[Interfaces Implemented] SET\_STATUS, SET\_EXCEPTION\_STATUS
			\item[Input] The ID of the car and the new status.
			\item[Output] The new status is set.
		\end{description}
	\item[CAR/Rides] \hfill
		\begin{description}[noitemsep]
			\item[Responsability] Retrieve the list of rides done with a specific car in a defined time range.
			\item[Interfaces Implemented] RIDE
			\item[Input] The ID of the car and a time range.
			\item[Output] The list of rides performed with that car in that time range.
		\end{description}
	\end{description}

	\item[BILLING\_SYSTEM] \hfill
	\begin{description}
		\item[Responsability] Manages all the fees.
	\item[BILL/Calculate] \hfill
		\begin{description}[noitemsep]
			\item[Responsability] Calculates the amount of a riding fee.
			\item[Interfaces Implemented] CALCULATE\_FEE
			\item[Input] The ID of the ride.
			\item[Output] The final fee, including eventual discounts or overprices.
		\end{description}
	\item[BILL/Pay] \hfill
		\begin{description}[noitemsep]
			\item[Responsability] Requires user to pay a specific bill.
			\item[Interfaces Implemented] PAY
			\item[Input] The ID of the user and the ID of the ride the bill refers to.
			\item[Output] The fee is paid.
		\end{description}

	\item[ISSUE\_MANAGER] \hfill
	\begin{description}
		\item[Responsability] Manages car's issues.
	\item[ISSUE/New] \hfill
		\begin{description}[noitemsep]
			\item[Responsability] Rise a new issue.
			\item[Interfaces Implemented] REPORT\_ISSUE
			\item[Input] ID of the car, ID of the user raising the issue, a title and a description of the issue.
			\item[Output] The ID of the reported issue.
		\end{description}
	\item[ISSUE/TakeCare] \hfill
		\begin{description}[noitemsep]
			\item[Responsability] Allows operators taking in charge of a particular issue.
			\item[Interfaces Implemented] TAKE\_CHARGE
			\item[Input] The ID of the issue, the ID of the operator.
			\item[Output] The operator is now responsable for the issue.
		\end{description}
	\item[ISSUE/Solve] \hfill
		\begin{description}[noitemsep]
			\item[Responsability] Allows operators to close an issue marking it as Solved. The system is resposible of allowing this operation only if it can confirm the issue has been solved.
			\item[Interfaces Implemented] SOLVE, VALIDATE\_SOLVE
			\item[Input] The ID of the issue, the ID of the operator.
			\item[Output] The issue is set as Solved, or set as Closed and another new issue is opened.
		\end{description}
	\item[ISSUE/GiveUp] \hfill
		\begin{description}[noitemsep]
			\item[Responsability] Allows operators to give up over an issue.
			\item[Interfaces Implemented] GIVE\_UP
			\item[Input] The ID of the issue, the ID of the operator.
			\item[Output] The issue is no more related to the specified operator and available for others to be took in charge.
		\end{description}
	\end{description}

\end{description}

{\color{blue} { Qui un Components View simile a quello a pagina 8 dell'esempio... possibilmente con un font un po' piu' grande che nell'esempio. \\ Non so bene se qui o in fondo alla sezione successiva. }}



\subsection{High-Level Architecture}

Up to this point we defined the logical components of the system in terms of implemented interfaces. We now proceed deploying them on the nodes identified in the first section of the analysis in order to end up with a complete, high-level logical architecture for our system.


\begin{description}
	\item[USER\_MANAGER] \hfill \\
	Considering that all its three functions are exposed as API by the server to the apps, the component is deployed on the server entirely.
		
	\item[LOCATION] \hfill \\
	Considering that all its three functions are exposed as API by the server, the component is deployed on the server entirely.

	\item[POSITION] \hfill \\
	Considering that all its three functions are exposed as API by the server, the component is deployed on the server entirely. {\color{red} {Define this better}}
	
	\item[BOOKING\_MANAGER] \hfill \\
	Considering that all its three functions are exposed as API by the server, the component is deployed on the server entirely. {\color{red} {What about EXPIRE?}}
	
	\item[CAR\_MANAGER] \hfill \\
	This component requires a deeper analysis, as its interfaces are exposed by different elements. In details:
	\begin{itemize}
		\item CAR/Unlock is exposed by Server to Apps
		\item CAR/Rides is exposed by Server to Apps
		\item CAR/ValidateLicense is exposed by Server to Cars
		\item CAR/Lock is exposed by Cars to Server
		\item CAR/TurnOff is exposed by Cars to Server
		\item CAR/Telemetry is exposed by Cars to Server
		\item CAR/SetStatus is exposed by Cars to Server
	\end{description}
	Thus we create two CAR\_MANAGER subcomponents:
	\begin{description}
		\item[REMOTE\_CAR\_MANAGER] \hfill \\ Comprises all the functions exposed by the server:
		\begin{itemize}
			\item CAR/Unlock
			\item CAR/Rides
			\item CAR/ValidateLicense
		\end{itemize}
		\item[LOCAL\_CAR\_MANAGER] \hfill \\ Comprises all the functions exposed by the cars:
		\begin{itemize}
			\item CAR/Lock
			\item CAR/TurnOff
			\item CAR/Telemetry
			\item CAR/SetStatus
		\end{itemize}
	\end{description}

	\item[BILLING\_SYSTEM] \hfill \\
	Considering that all its three functions are exposed as API by the server to the customer's apps, the component is deployed on the server entirely. {\color{red} {Recheck PAY : is it on server or on the app?}}

	\item[ISSUE\_MANAGER] \hfill \\
	Considering that all its three functions are exposed as API by the server to the customer's apps, the component is deployed on the server entirely. \\
	Indeed, ISSUE/Solve involves an interface that is exposed by the car: thus we define another subcomponent, \textbf{VALIDATE\_SOLVE}, to be deployed on the car to fulfill this function.


	\end{description}
\end{description}
{\color{blue} {Il graficino High-level Arch a pagina 8 del DD di esempio, per quanto inutile possa essere in questo caso.}}

\subsection{Deployement View}
{\color{blue} {Il graficino del deploy a pagina 10 del DD di esempio, se capisci che cosa diavolo sia.}}

\subsection{Runtime View}
{\color{blue} {Tanti bei Sequence Diagrams :P }}


\subsection{Selected Technologies}

Having defined the interface between the several functionality provided by the components, all possible communication technologies may have been choosen. Obviously some choices are more apt than others with the respect of latency, throughput, elegance of the design and other factors; however the whole design will remain intact whichever technology we choose.

Here we simply give some reasonable suggestions for protocols and technologies that seems us a better fit for the system we modeled.

\subsubsection{RESTful API}

The main server will expose its API in the most conventional way, using the classical HTTP/TCP/IP stack. In particular we provide RESTful interfaces. Model everything as an entity provides enough capabilities to actual implement the whole system while remaining constrainted to only the basic REST verb. This helped a lot in keeping the whole API scheme simple to understand and to implement.

Apps consume the REST interface provide by the server, however to implement the communication between the server and the app we will use the long polling strategy.

\subsubsection{MQTT}

As for now, most of the communication between the server and the car will happens using the PubSub protocol, while some special messages (like, for example, the ValidateSolve function) will stay on a more basic Client-Server approach.

 Between all the implementation of the PubSub protocol we chose to use MQTT for its low overhead, its QoS and because it is widely used in the industry.

\subsubsection{Distributing the main server}

{\color{red} {what about this?}}

\subsubsection{Database}

{\color{red} {what about this?}}



















\newpage
\section{OLD STUFF}

\subsection{Interfaces}

ns provided at registration time. \\ Fulfills \textbf{REG2} and \textbf{REG3}.
		\end{description}

	\item[LOGIN] Users can login to \pe.\hfill {\color{red}{double-check LOG4}}
		\begin{description}
			\item[LOGIN] Provided valid credentials, users are logged into the system and from now on they have the possibility to book cars, unlock cars, etc. \\ Fulfills \textbf{LOG1, LOG2, LOG3} and \textbf{LOG4}
		\end{description}

	\item[LOOKUP] Users can find cars nearby a given position, according to their search settings. \hfill  {\color{red}{missing LOOK4, LOOK5, LOOK6}}
		\begin{description}
			\item[LOOKUP] Logged users can retrieve a list of available cars according to their search settings.\\ Fulfills \textbf{LOOK1, LOOK2} and \textbf{LOOK3} 
		\end{description}

	\item[BOOK] Users can book a car for a short amount of time.\hfill {\color{red}{double-check. Missing BOOK1. EXPIRE added}}
		\begin{description}
			\item[BOOK] Logged users can reserve a car. \\ Fulfills \textbf{BOOK2?, BOOK3?, BOOK4?, BOOK5? }
			\item[EXPIRE] A booked car not unlocked after a system-defined period of time has passed is automatically unbooked and the user who booked it is fined. \\ Fulfills \textbf{BOOK6}
		\end{description}

	\item[UNBOOK] Users can decide to cancel a booking made before the expiration.\hfill {\color{red}{Missing UNBOOK1}}
		\begin{description}
			\item[UNBOOK] Logged users can cancel a reservation they made. \\ Fulfills \textbf{UNBOOK2}
		\end{description}

	\item[UNLOCK] Users can unlock the car they booked. \hfill {\color{red}{Missing UNLK4}}
		\begin{description}
			\item[UNLOCK] Logged users can unlock the car they booked. \\ Fulfills \textbf{UNLK1, UNLK2, UNLK5}
			\item[POSITION] {\color{red}{ The system must be able to locate the user. }}
			\item[UNLOCK\_CAR] The system can unlock a car. \\ Fulfills \textbf{UNLK3}
		\end{description}

	\item[RIDE] Users can drive to their destination. \hfill  {\color{red}{change RIDE def. Add PARK}}
		\begin{description}
			\item[RIDE] The system knows which user is driving which car at the present time and at any moment in the past. \\ Fulfills \textbf{RIDE1}
			\item[READ\_LICENSE] The system acquires information about the user's driving license and decides whether to let the user start the engine or not. \\ Fulfills \textbf{RIDE2, RIDE3}
			\item[SHOW\_INFORMATIONS] Once the ride started, the system shows to users basic informations such as nearby safe parking areas and nearby charging stations. \\ Fulfills \textbf{RIDE4, SAFE1, SAFE2, PWRS1, PWRS2}
			\item[PARK] Users can lock the car once they finished the ride and exited from the car. \\ Fulfills \textbf{RIDE5, SAFE3, SAFE4}
		\end{description}

	\item[SAFE\_AREAS] Users can locate safe parking areas. \hfill   {\color{red}{Overlaps with RIDE!}}
		\begin{description}
			\item[SHOW\_INFORMATIONS] Once the ride started, the system shows to users basic informations such as nearby safe parking areas and nearby charging stations. \\ Fulfills \textbf{RIDE4, SAFE1, SAFE2, PWRS1, PWRS2}
			\item[PARK] Users can lock the car once they finished the ride and exited from the car. \\ Fulfills \textbf{RIDE5, SAFE3, SAFE4}
		\end{description}

	\item[UNSAFE\_PARKING] The system reacts to an unsafe parking.\hfill {\color{red}{Missing UNSF1, UNSF2}}
		\begin{description}
			\item[TURN\_OFF] The system turns off a car left in an unsafe area. \\ Fulfills \textbf{UNSF3}
			\item[LOCK\_CAR] The system locks a car left parked in an unsafe area. \\ Fulfills \textbf{UNSF4}
		\end{description}

	\item[POWER\_STATION] Users can locate and use charging stations correctly. \hfill {\color{red}{Missing PWRS4}}
		\begin{description}
			\item[SHOW\_INFORMATIONS] Once the ride started, the system shows to users basic informations such as nearby safe parking areas and nearby charging stations. \\ Fulfills \textbf{RIDE4, SAFE1, SAFE2, PWRS1, PWRS2}
			\item[CAR\_PLUGGED] The system detects whenever a car is plugged to a power station. \\ Fulfills \textbf{PWRS3}
		\end{description}

	\item[CHARGE] At the end of the ride, the user is charged a fee.\hfill
		\begin{description}
			\item[CALCULATE\_FEE] The system calculates the total fee that the user must pay. \\ Fulfills \textbf{FEE1, FEE2, FEE3, FEE4, FEE5, FEE6}
			\item[SEND\_FEE] The system communicates to the user the total cost of the ride when the car gets locked. \\ Fulfills \textbf{FEE7}
		\end{description}

	\item[PAYMENT] Users can pay bills through the app.\hfill {\color{red}{Missing PAY2, PAY3, PAY4}}
		\begin{description}
			\item[PAY] The system asks users to pay ride fares. {\color{red}{No matching req. found .-.}}
			\item[SET\_PAYMENT\_METHOD] Users can set their preferred paying method. \\ Fulfills \textbf{PAY1}
		\end{description}

	\item[FIND\_ISSUES] The staff can locate cars that need their intervention. \hfill {\color{red}{Missing ISS2, ISS4}}
		\begin{description}
			\item[FIND\_ISSUE] Staff operators can locate issued cars that needs their intervantion. \\ Fulfills \textbf{ISS1, ISS3}
			\item[REPORT\_ISSUE] Users can report issues about a car.  {\color{red}{No matching req. found .-.}}
		\end{description}

	\item[SUPPORT] The staff can identify and solve car’s issues. \hfill {\color{red}{Add CONFIRM\_AVAILABILITY}}
		\begin{description}
			\item[TAKE\_CHARGE] The system allows operator to take charge of certain issues. \\ Fulfills \textbf{SUP1}
			\item[SET\_STATUS] Operators can change the statuses of issued cars. \\ Fulfills \textbf{SUP2, SUP4}
			\item[CONFIRM\_AVAILABILITY] The system checks if the issues marked as Solved are really related to cars that shows no issue. \\ Fulfills \textbf{SUP3}
		\end{description}
\end{description}

\subsection{Components}

Given the interface we identified in the previous section, we organize such interfaces into higher level components as follows.

Note that all the components are responsible of returning meaningful error messages in case of error.

\begin{description}
	\item[USER\_MANAGER] \hfill
	\begin{description}
		\item[Responsability] Manages the users.
	\item[USER/Register] \hfill
		\begin{description}
			\item[Responsability] Registers a new user into the system.
			\item[Input] Information from the user such as:
				\begin{itemize}
					\item Name
					\item Lastname
					\item Password
					\item Email
					\item License ID
					\item Credit card informations: credit card number, control code, expiry date, owner, etc.
				\end{itemize}
			\item[Output] The ID of the newly created user.
		\end{description}
	\item[USER/Login] \hfill
		\begin{description}
			\item[Responsability] Allows users to log into the system.
			\item[Input] Email (considered a unique user ID) and password.
			\item[Output] A session key, meaning that the user is logged into the system.
		\end{description}
	\end{description}
	
	\item[LOCATION] \hfill
	\begin{description}
		\item[Responsability] Locates elements, points and areas of interest around a specific coordinate. ``Search`` service for elements of interest.
	\item[LOCATION/AvailableCar] \hfill
		\begin{description}
			\item[Responsability] Retrives the position of available cars.
			\item[Input] Search parameters such as:
			\begin{itemize}
				\item Geographical coordinates of the center of the search range (latitude and longitude as provided by GPS sensors) 
				\item Maximum walking distance from the specified position
				\item Other search settings, like minimum battery level, etc.
			\end{itemize}
			\item[Output] A set of available cars matching the search parameters.
		\end{description}

	\item[LOCATION/Areas] \hfill
		\begin{description} 
			\item[Responsability] Retrives the position of areas of interest, such as power stations and safe parking areas.
			\item[Input] Geographical coordinates of the center of the search range (latitude and longitude as provided by GPS sensors) and a search radius.
			\item[Output] A set of areas of interest inside the circle of radius provided centered on the coordinates provided.
		\end{description}

	\item[LOCATION/IssuesCar] \hfill
		\begin{description}
			\item[Responsability] Retrives the position of cars with some issues.
			\item[Input] Search parameters such as:
			\begin{itemize}
				\item Geographical coordinates of the center of the search range (latitude and longitude as provided by GPS sensors) 
				\item Radius of the search
				\item Issue type, Exeption status, and other similar search settings.
			\end{itemize}
			\item[Output] A set of cars with issues matching the search parameters inside the circle of radius provided centered on the coordinates provided.
		\end{description}
	\end{description}

	\item[POSITION] \hfill
	\begin{description}
		\item[Responsability] Locatse elements of interest given their ID. ``Lookup`` service for elements of interest.
	\item[POSITION/Car] \hfill
		\begin{description}
			\item[Responsability] Retrieves the position of a specific car.
			\item[Input] The ID of the car.
			\item[Output] The coordinates of the car.
		\end{description}
	\item[POSITION/User] \hfill
		\begin{description}
			\item[Responsability] Retrieves the position of an user.
			\item[Input] The ID of the user.
			\item[Output] The coordinates of the user.
		\end{description}
	\item[POSITION/Areas] \hfill
		\begin{description}
			\item[Responsability]Retrieves the position of an area of interest.
			\item[Input] The ID of the area.
			\item[Output] The coordinates of the area as a set of boundary points.
		\end{description}
	\end{description}
	
	\item[BOOKING\_MANAGER] \hfill
	\begin{description}
		\item[Responsability] Manages reservations.
	\item[BOOKING/Book] \hfill
		\begin{description}
			\item[Responsability] Books one available car.
			\item[Input] The ID of the car and the ID of the user.
			\item[Output] The car is booked and the ID of the reservation is provided.
		\end{description}
	\item[BOOKING/Unbook] \hfill
		\begin{description}
			\item[Responsability] Removes a reservation.
			\item[Input] The ID of the user and the ID of the reservation.
			\item[Output] The reservation is cancelled.
		\end{description}
	\end{description}
	
	\item[CAR\_MANAGER] \hfill
	\begin{description}
		\item[Responsability] Manages the iteractions between users and cars.
	\item[CAR/Unlock] \hfill
		\begin{description}
			\item[Responsability] Unlocks the car.
			\item[Input] The ID of the car and the ID of the user asking to unlock.
			\item[Output] The car is unlocked.
		\end{description}
	\item[CAR/Lock] \hfill
		\begin{description}
			\item[Responsability] Locks the car.
			\item[Input] The ID of the car.
			\item[Output] The car is locked.
		\end{description}
	\item[CAR/TurnOff] \hfill
		\begin{description}
			\item[Responsability] Turns off the engine of a car.
			\item[Input] The ID of the car.
			\item[Output] The car is turned off.
		\end{description}
	\item[CAR/Telemetry] \hfill
		\begin{description}
			\item[Responsability] Retrieves real-time, updated informations about a car.
			\item[Input] The ID of the car.
			\item[Output] All the latest informations available about the car.
		\end{description}
	\end{description}

	\item[BILLING\_SYSTEM] \hfill
	\begin{description}
		\item[Responsability] Manages all the fees.
	\item[BILL/Calculate] \hfill
		\begin{description}
			\item[Responsability] Calculates the amount of a riding fee.
			\item[Input] The ID of the ride.
			\item[Output] The final fee, including eventual discounts or overprices.
		\end{description}
	\item[BILL/Pay] \hfill
		\begin{description}
			\item[Responsability] Requires user to pay a specific bill.
			\item[Input] The ID of the user and the ID of the ride the bill refers to.
			\item[Output] The request of payment.
		\end{description}
	\item[BILL/Collect] \hfill
		\begin{description}
			\item[Responsability] Complete the money transaction.
			\item[Input] Credit card informations of users and the ID of the bill they have to pay.
			\item[Output] The fee is paid.
		\end{description}
	\end{description}

	\item[ISSUE\_MANAGER] \hfill
	\begin{description}
		\item[Responsability] Manages car's issues.
	\item[ISSUE/New] \hfill
		\begin{description}
			\item[Responsability] Rise a new issue.
			\item[Input] ID of the car, ID of the user raising the issue, a title and a description of the issue.
			\item[Output] The ID of the reported issue.
		\end{description}
	\item[ISSUE/Modify] \hfill
		\begin{description}
			\item[Responsability] Allows operators to modify the statuses of some issues.\\ {\color{red}{ The operator may have fixed the issue, may decide that he is not capable to fix the issues or it may decide that the issues is \textbf{un-fixable}.}}
			\item[Input] The ID of the issue, the ID of the operator, the affected status and the new status value.
			\item[Output] The status of a issue is updated.
		\end{description}
	\item[ISSUE/TakeCare] \hfill
		\begin{description}
			\item[Responsability] Allows operators to take charge of a particular issue.
			\item[Input] The ID of the issue, the ID of the operator.
			\item[Output] The operator is now responsable for the issue.
		\end{description}
	\end{description}

\end{description}



\subsection{Logical Deploying}

\begin{description}
	\item[USER\_MANAGER] Logically deployed on the main server, while its API are invoked only by the apps.
	\item[LOCATION] Logically deployed on the main server. Its API are invoked by customer's apps (LOCATION/AvailableCar), by the fleet (LOCATION/Areas) and by the staff's apps (LOCATION/IssuesCar).
	\item[BOOKING\_MANAGER] Logically deployed on the main server, while its API are invoked by the customer's apps.

	\item[CAR\_MANAGER] Logically deployed in both the main server and the fleet. {\color{red}{Non dovremmo definire due componenti diversi, quello dell'auto e quello del server?}}
	\begin{description}	
		\item[CAR/Unlock], invoked by the apps, involves both the main server, that guarantees that the request is legitimate, and the selected car, that actually unlocks the door.
		\item[CAR/Lock] can be invoked by apps on directly by the server (in case of unsafe parking) and actuated by the cars.
		\item[CAR/TurnOff] is invoked by the server and actuated by the cars.
		\item[CAR/Telemetry]
 is invoked by the server and actuated by the cars.
	\end{description}

	\item[POSITION] Logically deployed partly on cars, partly on the users app and partly on a 3rd-part service. 
	\begin{description}
		\item[POSITION/Car] Deployed on the car itself and invoked, by the server on behalf of the customer's apps. 
		\item[POSITION/User] Deployed on the user application and invoked by the main server and, indirectly by the user itself (when the user ask to unlock a car he must be nearby the car itself). 
		\item[POSITION/Areas] Deployed on a 3rd-part service and invoked by the server on behalf of the cars.
	\end{description}

	\item[BILLING\_SYSTEM] Logically deployed partly on the main server, partly on the users' app and partly on a 3rd-part system. 
		\begin{description}
		\item[BILL/Calculate] Deployed and invoked by the server. 
		\item[BILL/Pay] Invoked by the server but executed on the users application
		\item[BILL/Collect] is invoked on the server but actually executed in a 3rd part system.
		\end{description}

	\item[ISSUE\_MANAGER] This manager is deployed in the main server and it functionality are invoked by the apps, both customer's apps to report issues and staff's app to find issued cars.
\end{description}
	
\subsection{Deploy}

At this point we have understood, logically, where each component should be deployed and what part of the system invoke what functionality.

Now we are going to physically deploy all the functionality in the correct part of the system and we are going to define a communication mechanism between those functionality.

The interface between the several functionality provided by the components are already defined, from this point on we are going to pick a particular technology only for the sake of simplicity, all the possible communication technology may have been choosen. Obviously some choices are more apt than others with the respect of latency, throughput, elegance of the design and other factors; however the whole design will remain intact whichever technology we choose.

\subsection{Selected Patterns and Technologies}

The reason for the technology choice we made are expressed in this section.

The main server will expose its API in the most conventional possible way, using the classical HTTP/TCP/IP stack. In particular we focus ourselves in provide RESTfull interfaces. Model everything as an entity will provide enough capabilities to actuall implement the whole system while remaining constrainted to only the basic REST verb will help in keeping the whole API simple.

The users app will consume the REST interface provide by the server, however to implement the communication between the server and the app we will use the long polling strategy.

We believe that the car will need to communicate very often a lot of valuable information to the main server and in order to achieve high troughput and low latency we believe that the PubSub protocol is the most apt. Moreover, the PubSub protocol is pretty natural in this scenario, having the car publish messages about it own status and having the the server subscribe to those messages. Also most of the communication between the server and the car will happens using the PubSub protocol. Between all the implementation of the PubSub protocol we chose to use MQTT for its low overhead, its QoS and because it is widely used in the industry.



\subsection{The rest}

The high lever architecture of the system is made up of three main elements:
\begin{itemize}[noitemsep]
	\item The main server
	\item The mobile apps (user apps and staff apps alike)
	\item The car's onboard system
\end{itemize}
On top of these elements, there is a number of external services the servers interacts with in order to provide functionalities to the apps or to the cars' system.

\begin{figure}[H]
	\centering
	\includegraphics[width=1\textwidth]{proposed_system.png}
	\caption{High level overview of the architecture}
\end{figure}	

At this stage, the system's architecture is clearly two-tier:
\begin{itemize}[noitemsep]
	\item Tier 1, the main server, handles the application logic and data management.
	\item Tier 2, comprising mobile apps and cars, hosts the User Interface.
\end{itemize}

The system blends three different interaction models for each of the three pair of components interacting. In detail, we designed:
\begin{itemize}
	\item a pure \textbf{Client Server} approach when the main server interacts with the apps (customer's apps and staff's apps alike)
	\item a \textbf{Service Oriented} communication model between the main server interacting with external services like linf.io, truelicence.it, stripe.com etc.
	\item a \textbf{Publisher Subscriber} model between the main server and the cars' systems.
\end{itemize}

In subsequent sections we will provide more details about these components.

\subsection{Component View}

\subsection{Deployement View}

\subsection{Runtime View}
	\textit{ [Includes sequence diagrams to show how components interact to accomplish specific use cases]}

\subsection{Component Interfaces}

\subsection{Architectural Styles and Patterns}
	\textit{ [Explain patterns used above]}

\subsection{Other Design Decisions}


\newpage
\section{Algorithm Design}
	\textit{ [Definition of critical sections of code] }


\newpage
\section{User Interface Design}

\subsection{Mockups}
Mockups have already been included in the RASD (section 3.3: Non Functional Requirements) 

\subsection{UX Diagrams}

\begin{figure}[H]
	\centering
	\includegraphics[width=0.88\textwidth]{UML/UXDiagramPublicApp.png}
	\caption{UX Diagram of the interface of customer's application}
\end{figure}

\begin{figure}[H]
	\centering
	\includegraphics[width=0.88\textwidth]{UML/UXDiagramStaffApp.png}
	\caption{UX Diagram of the interface of staff's application}
\end{figure}	

\subsection{BCE Diagrams}

\begin{figure}[H]
	\centering
	\includegraphics[width=1\textwidth]{UML/BCEDiagramPublicApp.png}
	\caption{BCE Diagram of the interface of customer's application}
\end{figure}	

\begin{figure}[H]
	\centering
	\includegraphics[width=1\textwidth]{UML/BCEDiagramStaffApp.png}
	\caption{BCE Diagram of the interface of staff's application}
\end{figure}	
	

\newpage
\section{Requirements Traceability}

{\color{red} {Lo stesso lavoro fatto nel definire le interfacce, ma a rovescio. Non piu' Interfaccia -> Req soddisfatti, ma Req-> interfaccia che lo soddisfa.}}

Considering we designed the system using a bottom-up approach, designed components maps in a straightforward way to the goals specified in the RASD. Hoever we provide an explicit mapping of the two.

 \begin{description}
 	\item[REGISTRATION] Users can register to \pe.
	\begin{itemize}
		\item Server: RegistrationController
		\item Customer's App: RegistrationView (?)
	\end{itemize}

	\item[LOGIN] Users can login to \pe.
	\begin{itemize}
		\item Server: LoginController
		\item Customer's App: LoginView (?)
		\item Staff's App: LoginView (?)
	\end{itemize}

 	\item[LOOKUP] Users can find cars nearby a given position, it could be its position or a point in the map.
	\begin{itemize}
		\item Server: CarsLocation
		\item Customer's App: CarsView (?)
	\end{itemize}

 	\item[BOOK] Users can book a car for a short amount of time.
	\begin{itemize}
		\item Server: BookingController
		\item Customer's App: BookingView (?)
	\end{itemize}

 	\item[UNLOCK] When users are in proximity of the car they booked, the system can unlock it.
	\begin{itemize}
		\item Server: CarLockController
		\item Car: LockController
		\item Customer's App: CarNearby (----------------- SEE ISSUE #14)
	\end{itemize}

	\item[RIDE] Users can drive to their destination.
	\begin{itemize}
		\item Server: AuthDriver
		\item Car: LicenceScanner, EngineController, MQTT publishers, CarGUI
	\end{itemize}

	\item[SAFE\_AREAS] Users can locate safe parking areas.
	\begin{itemize}
		\item Server: AreasLocation
		\item External Services: linf.io
		\item Car: MQTT publishers, CarGUI
	\end{itemize}

	\item[UNSAFE\_PARKING] The system must react to an unsafe parking.
	\begin{itemize}
		\item Server: UnsafeParkingController, IssuesController (?)
		\item Car: EngineController, LockController (MQTT publishers too?)
	\end{itemize}

	\item[POWER\_STATIONS] Users can locate charging stations.
	\begin{itemize}
		\item Server: AreasLocation
		\item External Services: linf.io
		\item Car: CarGUI
	\end{itemize}

	\item[CHARGE] At the end of the ride, users are charged a fee.
	\begin{itemize}
		\item Server: BillingController
		\item Customer's App: PaymentDetailsView
	\end{itemize}

	\item[PAYMENTS] Users can pay bills through the app.
	\begin{itemize}
		\item Server: PaymentController
		\item External Services: stripe.com
		\item Customer's App: PaymentDetailsView
	\end{itemize}

	\item[FIND\_ISSUES] The staff can locate cars that need their intervention.
	\begin{itemize}
		\item Server: IssuesLocation
		\item Staff's App: IssuesView (?)
	\end{itemize}

	\item[SUPPORT] The staff can identify and solve car's issues.
	\begin{itemize}
		\item Server: IssuesController
		\item Staff's App: IssueDetailsView (?)
	\end{itemize}

	\item[FINES] The system can provide enough details for the staff to manage correctly the fines they receive from local authorities.
	\begin{itemize}
		\item Server: FindDriverController (?)
		\item Staff's App: FindDriverView (?)
	\end{itemize}

 \end{description}


\newpage
\section{Conclusions}

\subsection{Tools used}
During the development of this document we used the following tools:
\begin{itemize}
	\item \textbf{Github} to version control the project
	\item \textbf{\LaTeX} on TeXworks to redact this document
	\item \textbf{www.draw.io} to draw UML graphs
	\item \textbf{Gimp v.2.8} to mockup the application
	\item \textbf{LibreOffice Draw} to draw the system's overview at section 2.1
\end{itemize}

\subsection{Hours of work}
\begin{itemize}
	\item SZ: 1h on 30/11
	\item SM: 5h on 2/12
	\item SZ: 5h on 2/12
	SZ: 3h of work on 4/12 (RASD review)

	\item SZ: 3h on 5/12
	\item SZ: 4h on 6/12
	\item SZ: 7h on 7/12
\end{itemize}


\end{document}  

%%%%%%%%%% PARTE VECCHIA QUA SOTTO %%%%%%%%%%%%%%%%%


\subsection{Proposed System}

The proposed system features a client-server architecture, so it is divided into two parts: a frontend app for smartphones, which allows the users to use the service, and a backend system wich deals with all the operations and coordinates them. The backend also interact with the cars, that can be seen as a third part of the system.

\begin{figure}[H]
	\centering
	\includegraphics[width=1\textwidth]{proposed_system.png}
	\caption{Description of the proposed system.}
\end{figure}

\subsubsection{App Frontend}
The frontend is a thin app that relies on the smartphone's internet connections in order to work. Almost no operations can be performed with the app alone: all transactions are sent to the main server first, then processed and the result sent back to the app.

The app can be classified as a thin client.

\subsubsection{Centralized Backend}
The backend is the core of the system. Being able to process a lot of parallel operations, it can deal with all the requests coming from che clients ina reasonable amount of time (see Non Functionals Requirements). The backend is based on an MVC architecture and a REST API.

\subsubsection{Car's Onboard system}
The cars are equipped with an onboard system that monitors the status of the car, its location, and can send all the necessary informations to the main server. There won't be direct interactions between the car and the user's app.


