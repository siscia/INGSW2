% !TEX TS-program = pdflatex
% !TEX encoding = UTF-8 Unicode

% This is a simple template for a LaTeX document using the "article" class.
% See "book", "report", "letter" for other types of document.

\documentclass[11pt]{article} % use larger type; default would be 10pt.
\setcounter{secnumdepth}{2}

\usepackage{paralist} % very flexible & customisable lists (eg. enumerate/itemize, etc.)

\usepackage[utf8]{inputenc} % set input encoding (not needed with XeLaTeX)
\usepackage{float} % to place float images correctly
\usepackage{color} % to color text
\usepackage{enumitem} % for lists
\usepackage{subfigure} % for mockups
\usepackage[font={it}]{caption} % for captions

%%% Examples of Article customizations
% These packages are optional, depending whether you want the features they provide.
% See the LaTeX Companion or other references for full information.

%%% PAGE DIMENSIONS
\usepackage{geometry} % to change the page dimensions
\geometry{a4paper} % or letterpaper (US) or a5paper or....
% \geometry{margin=2in} % for example, change the margins to 2 inches all round
% \geometry{landscape} % set up the page for landscape
%   read geometry.pdf for detailed page layout information

\usepackage{graphicx} % support the \includegraphics command and options

% \usepackage[parfill]{parskip} % Activate to begin paragraphs with an empty line rather than an indent

\usepackage{listings}
\usepackage{color}
 
\definecolor{codegreen}{rgb}{0,0.4,0}
\definecolor{codegray}{rgb}{0.5,0.5,0.5}
\definecolor{codepurple}{rgb}{0.58,0,0.82}
 
\lstdefinestyle{mystyle}{ 
    commentstyle=\color{magenta},
    keywordstyle=\color{blue}\bfseries,
    numberstyle=\tiny\color{codegray},
    stringstyle=\color{codepurple},
    basicstyle=\footnotesize,
    breakatwhitespace=false,         
    breaklines=true,                 
    captionpos=b,                    
    keepspaces=true,                 
    numbers=left,                    
    numbersep=5pt,                  
    showspaces=false,                
    showstringspaces=false,
    showtabs=false,                  
    tabsize=2,
   emph={self}, 
   emphstyle=\color{blue}, 
   emph={[2] BookingManager, FineManager}, 
   emphstyle=[2]\color{codegreen}\bfseries, 
   emph={[3]__init__, newBook, removeReservation, getReservation, manageExpired, min_heap, pop, now, timedelta, expireReservation}, 
   emphstyle=[3]\color{codegreen}, 
}
 
\lstset{style=mystyle}





%%% PACKAGES
\usepackage{booktabs} % for much better looking tables
\usepackage{array} % for better arrays (eg matrices) in maths
%\usepackage{paralist} % very flexible & customisable lists (eg. enumerate/itemize, etc.)
\usepackage{verbatim} % adds environment for commenting out blocks of text & for better verbatim
\usepackage{subfig} % make it possible to include more than one captioned figure/table in a single float
% These packages are all incorporated in the memoir class to one degree or another...

%%% HEADERS & FOOTERS
\usepackage{fancyhdr} % This should be set AFTER setting up the page geometry
\pagestyle{fancy} % options: empty , plain , fancy
\renewcommand{\headrulewidth}{0pt} % customise the layout...
\lhead{}\chead{}\rhead{}
\lfoot{}\cfoot{\thepage}\rfoot{}

%%% SECTION TITLE APPEARANCE
\usepackage{sectsty}
\allsectionsfont{\sffamily\mdseries\upshape} % (See the fntguide.pdf for font help)
% (This matches ConTeXt defaults)

%%% ToC (table of contents) APPEARANCE
\usepackage[nottoc,notlof,notlot]{tocbibind} % Put the bibliography in the ToC
\usepackage[titles,subfigure]{tocloft} % Alter the style of the Table of Contents
\renewcommand{\cftsecfont}{\rmfamily\mdseries\upshape}
\renewcommand{\cftsecpagefont}{\rmfamily\mdseries\upshape} % No bold!



\usepackage{listings}
\usepackage{pxfonts}

%%% END Article customizations

%%% The "real" document content comes below...




\title{Apache OFBiz: Code Inspection of Selected Classes \\ {\Large Version 1.0}}
\author{Simone Mosciatti \& Sara Zanzottera}


\begin{document}
\maketitle
\newpage
\tableofcontents
\newpage

\section{Introduction}
This document described the results of a peer review of a few selected classes taken from the Apache OFBiz Project (official website: https://ofbiz.apache.org), an open source product for the automation of enterprise processes that includes framework components and business applications for ERP (Enterprise Resource Planning), CRM (Customer Relationship Management) and other business-oriented functionalities.
 
The code inspection has been carried out following the guidelines described in the reference document provided and consulting the official documentation \\ (https://ofbiz.apache.org/documentation.html) 

\subsection{Reference Documents}
\begin{itemize}
	\item \textit{Code Inspection Assignement Task Description.pdf}
	\item Official Documentation of the project at https://ofbiz.apache.org/documentation.html
	\item Source Code of the project at http://mirror.nohup.it/apache/ofbiz/apache-ofbiz-16.11.01.zip
	\item \textit{Brutish Programming} at 
http://users.csc.calpoly.edu/~jdalbey/SWE/CodeSmells/bonehead.html
\end{itemize}


\section{Assigned Classes}

The class assigned to our group is:  
\newline

\texttt{apache-ofbiz-16.11.01/framework/minilang/src/main/java/org/apache/ofbiz/ 
minilang/MiniLangUtil.java}

\hfill\\

This class has a role

 <elaborate on the func-
tional role you have identified for the class cluster that was assigned
to you, also, elaborate on how you managed to understand this role
and provide the necessary evidence, e.g., javadoc, diagrams, etc.>


\section{Issues Found}
Issues are listed first concerning the class as a whole, and then method by method.

\subsection{MiniLangUtil class}

Positive aspects:
\begin{itemize}[noitemsep]
	\item Naming conventions are respected thoughout the whole class.
	\item Indentation is consistently done using 4 spaces.
	\item Braces are used consistently accrding to the "Kernighan and Ritchie” style, and no single line blocks are left without brackets.
	\item Line alignement is consistent and tidy.
	\item A few comments are used in the class (see subsection 3.9: isConstantPlusExpressionAttribute)
	
\end{itemize}

Issues and negative aspects:
\begin{itemize}
	\item File organization shows some issues. A lot of lines exceed the 80-char limits, and a few of them (line 111, 152 and 172) exceed the 120-char limit too. In detail, line 111 breaks after char 152 (the total line size is 260 chars), while line 152 reaches the 138 chars and line 172 reaches 124 chars.

	\item Concerning Java Source Files organization, a peculiar issue has been found. This source file actually contains two classes: the main class, defined \texttt{ public final class MiniLangUtil} and a second, tiny class, defined at the end of the file as an internal, but public class, defined as \texttt{public static class PlainString}.
The second class is empty.

We undestand that creating a specific source file for an empty class like \textt{PlainString}
 can be overkill, but anyway this solution looks awkward. A public class in general has no reason for being internal to \textt{MiniLangUtil}: this solution simply makes difficult for developerd to locate it. In order to clarify the issue, we should understand better if the class is even needed: it is referenced into \texttt{MiniLangUtil} twice (line 165 and line 213), but it is not evident why the developer felt the need to create this class instead of using a plain \texttt{String} class.

	\item Javadoc is inconsistently detailed, and lacks fundamental informations for some methods (see next subsections).

	\item Static class variables are not perfectly ordered (some private variables comes before package level variables)

	\item Definition of \texttt{SCRIPT\_PREFIXES} is cumbersome: it may be required, but there may also be some more elegant solutions to provide the required values for this constant instead of building the string inplace.

	\item PUNTO 33: PERCHÈ?

\end{itemize}

\subsection{containsScript (Line 79)}
\begin{itemize}
	\item Line 80: \texttt{str.lenght()} is called, but \texttt{str} is not guaranteed to be not null. 
	\item The method's Javadoc does not says that the behavior of the method in case of null input is a NullPointerException (lacks a @throw statement), while it should be the case.
\end{itemize}

\subsection{autoCorrectOn (Line 96)}
No issues found.

\subsection{callMethod (Line 111)}
\begin{itemize}
	\item This method's Javadoc lacks a lot of informations, starting from parameter's descriptions.
	\item One of the parameters (\texttt{retFieldFma}) has a name that becomes understandable only reading the whole method. It should at least be described in the Javadoc.
	\item The method has 7 input arguments, but only \texttt{parameters} is null checked (line 114). Except from \texttt{parameters} and \texttt{retFieldFma}, other arguments used only as arguments for other method's calls. 
	\item Line 133: \texttt{retFieldFma.isEmpty()} is called but \texttt{retFieldFma} is not guaranteed to be not null. 
	\item Line 136: each \texttt{Exception} is caught and changed into a \texttt{MiniLangRuntimeException}. This can be reasonable considering the context (the code simulates a method call), but looks like a code smell.
\end{itemize}

\subsection{convertType (Line 152)}
\begin{itemize}
	\item This method's Javadoc lacks a lot of informations, starting from parameter's description.
	\item Line 151: \texttt{@SuppressWarnings} refers to line 172, where an unchecked cast to \texttt{Converter<Object, Object>} is performed. Considering that \texttt{Converters} is a class that belongs to the project (see, at line 38, the \texttt{import} statement) this is probably due to a misunderstanding between developers that are in charge of these classes. \texttt{Converters} should be modified to provide a \texttt{Converter<Object, Object>} in output.
	\item \texttt{==} is often used in place of \texttt{.equals()} to compare objects: see lines 153, 159, 165, 169. These \texttt{==} comparisons should be safe, as they are used only for comparing classes, but as a general rule, it is desirable to use \texttt{.equals()} in any case.
	\item This method contains a snippet of code that could be moved elsewhere: lines 182 to 184 looks like input validation for the subsequent call to \texttt{localizedConverter.convert(obj, locale, timeZone, format)}, that would be better located into that method itself.
\end{itemize}

\subsection{flagDocumentAsCorrected (Line 194)}
\begin{itemize}
	\item This method's Javadoc lacks a lot of informations, starting from parameter's description.
	\item Line 195: \texttt{element} is not guaranteed to be not null, but a call to \texttt{element.getOwnerDocument()} is issued.
\end{itemize}

\subsection{getObjectClassForConversion (Line 211)}
\begin{itemize}
	\item This method's Javadoc, even being a little more detailed that the previous ones, lacks parameter's description and could be a little more detailed about the behavior or the method concerning \texttt{Map, List} and \texttt{Set} classes.
\end{itemize}

\subsection{isConstantAttribute (Line 235)}
\begin{itemize}
	\item Line 236: A call on \texttt{attributeValue.lenght()} is performed, but \texttt{attributeValue} is not guaranteed to be not null. This leads to a peculiar behavior: if \texttt{attributeValue == ""}, it returns \texttt{true}, while if \texttt{attributeValue == NULL}, it throws a NullPointerException (not described in the Javadoc).
\end{itemize}

\subsection{isConstantPlusExpressionAttribute (Line 251)}
\begin{itemize}
	\item Line 252: A call on \texttt{attributeValue.lenght()} is performed, but \texttt{attributeValue} is not guaranteed to be not null. This leads to a peculiar behavior: if \texttt{attributeValue == ""}, it returns \texttt{true}, while if \texttt{attributeValue == NULL}, it throws a NullPointerException (not described in the Javadoc).
	\item Comments are used in a cryptic way. The statement does not explain why concatenated expressions should be avoided. It does not even contains \texttt{TODO} or \texttt{FIXME} tags.
\end{itemize}

\subsection{isDocumentAutoCorrected (Line 273)}
\begin{itemize}
	\item Line 174: a call to \texttt{document.getUserData()} is issued, but \texttt{document} is not guaranteed to be not null. The consequent NullPointerException is not even stated in the Javadoc.
\end{itemize}

\subsection{writeMiniLangDocument (Line 284)}
\begin{itemize}
	\item This method's Javadoc lacks a lot of informations, starting from parameter's descriptions.
	\item Line 289: the path \texttt{"component://minilang/config/MiniLang.xslt"} is hardcoded. It should be at least moved into a private constant or made configurable by putting it in a config file.
	\item Exceptions management in this method is poor. Twice (line 293 and 310) there is a \texttt{catch(Exception e)} statement that replaces the original exception with a generic error message.
	\item Line 307: \texttt{xmlURL.getFile()} is called, but \texttt{xmlURL} is not guaranteed to be not null. In addition, this statement is put into a \texttt{try \{ \} catch(Exception e)} block that can completely mask this trivial issue.
	\item Line 311: \texttt{xmlURL} is concatenated to the error message string without even checking it for being not null. This can raise a new Exception inside the \texttt{catch} block.
\end{itemize}





\section{Conclusions}

\subsection{Tools used}
During the development of this document we used the following tools:
\begin{itemize}
	\item \textbf{Github} to version control the project
	\item \textbf{\LaTeX} on TeXworks to redact this document
\end{itemize}

\subsection{Hours of work}
\begin{itemize}
	\item SZ: 6h on 11/01
\end{itemize}


\end{document}  
